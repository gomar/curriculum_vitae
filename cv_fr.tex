\documentclass[10pt]{article}

\usepackage{mycv}

% ================
% = THE DOCUMENT =
% ================
\begin{document}
\begin{tabular}[ht]{clr}
	\multicolumn{2}{l}{\LARGE Adrien Gomar} & \multirow{6}{*}{} \\
	& & \\
	\fasymbol{"F041} & 1, allée Charles Malpel, 31300 Toulouse, France \\
	\fasymbol{"F095} & (+33) 6 14 32 24 24 \\
	\fasymbol{"F0E0} & \url{adrien.gomar@gmail.com} \\
	% \fasymbol{"F08C} & \url{http://fr.linkedin.com/in/adriengomar} \\
	\multicolumn{2}{l}{\url{http://cerfacs.fr/~gomar}} & 
	\phantom{aaaaaaaaaaaaaaaaaaaaaaaaaaaaa} {}
\end{tabular}

% ==============
% = EXPERIENCE =
% ==============
\mysec{Experience}
\begin{mytable}
\tblentry{2011 \\ à ce jour}{

{\headingfont CERFACS, Toulouse, 
Thèse de doctorat Arts et Métiers ParisTech}

"Aéroélasticité des hélices contra-rotatives non carénées":\newline
- extension d’une méthode d’ordre réduit pour la simulation instationnaire de turbomachines, \newline
- analyse des limites et amélioration de la méthode en vue de son intégration dans l’industrie. \newline
Co-développeur de la librairie de post-traitement Antares (www.cerfacs.fr/antares). Formation à son utilisation et communication (développement d’un site web, présentations aux industriels).
Vacations en école d’ingénieur (ISAE) \spacerows \\}

\tblentry{2010 \\ (6 mois)}{

{\headingfont CERFACS, Toulouse, 
Stage de fin d’étude}

Simulation aux grandes échelles d'un compresseur transsonique. 
Publication des résultats pour l'European Turbomachinery Conference (mars 2011) \spacerows \\}

\tblentry{2009 \\ (3 mois)}{

{\headingfont LIEBHERR Aerospace, Toulouse, 
Stage ingénieur}

Simulation stationnaire d’un compresseur centrifuge 
pour différents points de fonctionnement à l’aide d'un 
code de calcul industriel (\textit{elsA}). 
Comparaison des résultats avec un autre code de calcul (Fine/Turbo) \spacerows \\}

\tblentry{2008-2009 \\ (projet étudiant long)}{

{\headingfont Institut Supérieur de l’Aéronautique et de l’Espace (ISAE), Toulouse, 
Projet d'initiative personnel}

Mise en place d’un dispositif expérimental pour analyser un extracteur d’air d’avion. 
Simulations numériques stationnaires. 
Comparaison des résultats expérimentaux et numériques. 
Présentation des résultats à l'industriel (Technofan) \\}

\multicolumn{2}{c}{\phantom{a}} \\

\tblentry{2007-2009 \\ (2.5 ans)}{

{\headingfont Mac Donald's, Toulouse, Travail étudiant}

Management d'une équipe (15 personnes), 
amélioration des processus et formations aux différents postes}
\end{mytable}

% =============
% = EDUCATION =
% =============
\mysec{Education}
\begin{mytable}
\tblentry{2011 \\ à ce jour}{

{\headingfont Arts et métiers Paris Tech, Paris, 
France}

Préparation d’une thèse de doctorat Arts et Métiers ParisTech. 
Formations réalisées dans le cadre de la thèse: 
gestion de conflit, gestion de projet et pilotage d’équipe \spacerows \\}

\tblentry{2007-2010}{

{\headingfont Institut Supérieur de l'Aéronautique et de l'Espace (ISAE), 
Toulouse, France}

Diplôme d’ingénieur. 
Spécialisation en turbomachines et en mécanique des fluides numériques.
Responsable du club musique de l’ISAE (ENSICA), membre de l’équipe de rugby \spacerows \\}

\tblentry{2005-2007}{

{\headingfont Lycée militaire de St Cyr, St Cyr l'école, 
France}

Classes préparatoires aux grandes écoles, 
Spécialité Physique et Science de l’ingénieur}
\end{mytable}

% =============================
% = COMPETENCES INFORMATIQUES =
% =============================
\mysec{Compétences informatiques}
\begin{mytable}
\tblentry{Programmation}{

Maîtrise de la programmation orientée objet 
et de la gestion de projets informatiques (Antares) \spacerows  \\}

\tblentry{Langages}{

Python, NumPy, SciPy, Git, Unix, LaTeX, Gnuplot, HTML, CSS et C/C++}
\end{mytable}

% =============================
% = COMPETENCES LINGUISTIQUES =
% =============================
\mysec{Compétences linguistiques}
\begin{mytable}
\tblentry{Anglais}{

Niveau avancé: rédaction de compte-rendus et publications en rapport avec mon travail, présentation techniques et animation de réunions. Capable d’entretenir une conversation professionnelle et privé \spacerows \\}

\tblentry{Allemand}{

Bonnes bases (vécu 5 ans en pays germanophones: 1996-1998, 2000-2003), 
à réactualiser}
\end{mytable}

% =====================
% = CENTRE D INTERETS =
% =====================
\mysec{Centres d’intérêts}
\begin{mytable}
\tblentry{Culture}{

Pratique de la basse en groupe, composition et enregistrement. 
Nombreux voyages à l’étranger \spacerows \\}

\tblentry{Sport}{

Pratique de la course à pied et du rugby, 
actuellement membre de l’équipe de Météo-France (demi de mêlée)}
\end{mytable}

\mysec{Publications}
\begin{center}
	\begin{minipage}[ht]{0.9\textwidth}
		\nocite{Guedeney2013317}
		\nocite{Sicot2012uq}
		\bibliography{publication}
	\end{minipage}
\end{center}


\bibliographystyle{plain}
\end{document}