\documentclass[10pt]{article}

\usepackage{mycl}

% ================
% = THE DOCUMENT =
% ================
\begin{document}
\begin{flushright}
	1 allée Charles Malpel, Appt. A72 \\
	31300 Toulouse, France \\
	(+33) 6 14 32 24 24 \\
\end{flushright}

\begin{flushright}
September 8\textsuperscript{th}, 2013
\end{flushright}

\begin{flushleft}
Dyson Ltd \\
Tetbury Hill, Malmesbury \\
Wiltshire, UK
\end{flushleft}

\noindent
Dear Sir/Madam,
\newline

\noindent
I am writing to apply for a position as a Fluid Dynamics Engineer in the Aero-Acoustic Research team as advertised on your website, starting in March 2014. It is my feeling that my academic background, in addition to my work experience, will permit me to excel in a workgroup at Dyson. I was particularly excited to read about this opportunity offered by your company, as I am strongly interested in working in fluid dynamics engineering.
\newline

\noindent
I am currently a PhD candidate at CERFACS in Toulouse where I am investigating efficient algorithms to compute unsteadiness in turbomachines using CFD. Through my PhD work, I have theoretically analysed the so-called algorithm on simple and analytical problems to help understand its primitive behaviour. Then I have found innovative solutions to improve the algorithm and written a major publication on it with my colleagues in a leading journal in my discipline (Journal of Computational Physics). Finally, I have applied this algorithm to compute the aeroelastic characteristics of an industrial contra-rotating open rotor.
\newline

\noindent
While completing my MS in Aeronautical Engineering at ISAE (Institut Superieur de l’Aeronautique et de l’Espace), I acquired a multi-disciplinary education that has been completed with a strong scientific background obtained through my PhD experience. As can be seen from my CV, I have specific experience in the fields of fluid dynamics, CFD and thermodynamics. I have done several school projects within a wide range of disciplines as for instance electronic projects using Matlab or experimental fluid dynamics projects on small wind tunnels. Moreover, I have done a long student project on the experimental analysis of a commercial airplane extract fan. In this project, with my team, I have set up an experimental approach to measure radial distribution profiles of the pressure and the velocity fields. This has been done using a five-hole probe and the LabVIEW measurement software. Moreover, we have run CFD computations of the same geometry using NUMECA-Fine/Turbo to validate both the experimental approach and the CFD results. This gave me a deep understanding of the problems that may arise while setting up an experiment and push me to always try to validate my CFD results against experimental data. In addition, my education has provided me with an abundance of opportunities to define and solve problems, test my spirit of initiative and strengthen my communicative abilities. Working on several team projects has given me the chance to use my scientific background, learn to be more autonomous and discover a real taste for teamwork. This has permitted me to put into practice my theoretical knowledge to solve concrete industrial issues.
\newline

\noindent
Dyson is a dynamic and attractive organization looking for people having sound technical knowledge but, more importantly, the spirit of initiative and the taste for teamwork necessary to meet its standards of quality. In addition to my education, I have a varied set of experiences that have given me the skills to qualify for this position. During my PhD, my communicative, initiative and problem solving skills were sharpened. In fact, I have given several technical presentations to expert and non-expert audience and written two publications on leading journals in my discipline. Moreover, as I faced a post-processing issue for my computations, I have decided to develop my own post-processing software called Antares (www.cerfacs.fr/antares) to process large databases. I have also written a documentation and a website for this tool in addition to promoting it among industry and training users. I am proud that it is currently used worldwide. Whether in an office as a PhD candidate or on the rugby field as a team member, I have gained a strong spirit of initiative and a commitment to teamwork. My leadership skills have helped me develop the ability to collaborate with a diverse group of people in order to succeed in a project, which is necessary to work at Dyson.
\newline

\noindent
The idea of contributing to future research areas and innovation at Dyson would be very exciting for me. I am truly convinced that I can successfully conquer any task or challenge associated with this position. I feel that my technical training and my work experience will permit me to collaborate effectively with my future colleagues.
I will be available from March 2014. Please do not hesitate to contact me if I can provide further information. I am available for an interview at your convenience.  I look forward to hearing from you.
\newline

\noindent
Thanks for your consideration.
\newline

\begin{flushright}
Sincerely, \\
Adrien Gomar
\end{flushright}
\end{document}