\documentclass[10pt]{article}

\usepackage{mycl}

% ================
% = THE DOCUMENT =
% ================
\begin{document}
\begin{flushright}
	1 allée Charles Malpel, Appt. A72 \\
	31300 Toulouse, France \\
	(+33) 6 14 32 24 24 \\
\end{flushright}

\begin{flushleft}
Dyson Ltd \\
Tetbury Hill, Malmesbury \\
Wiltshire, UK
\end{flushleft}

\begin{flushright}
September 8\textsuperscript{th}, 2013
\end{flushright}

\noindent
Dear Sir or Madam,
\newline

\noindent
I am most interested in your advertisement "Fluid
Dynamics Engineer" (ref: PS5860) which I read in your website.
\newline

\noindent
Through my experiences, I have worked with Computational Fluid Dynamics
softwares to predict the flow physic that develops in turbomachines. From centrifugal 
compressors to open rotors through axial turbines, I have simulated 
almost all kind of turbomachines.
This have given me both the knowledge of 
fluid dynamics problems that may arise in such components and 
the hands-on experience in CFD, which are required for this position.
\newline

\noindent
In my long student project,
I have set up an experimental approach to measure the radial distribution
of thermodynamics variables using a five hole probe and, in parallel, the same
geometry has been computed using an industrial CFD code (Fine/Turbo). The comparison
of the experimental and computational results gave me a deep 
understanding of the fact that neither the experimental nor the computational
results are correct, both give information that are complementary.
This idea drove me to compare, when feasible, my results with
experimental data and to stand back from my computations to critically
analysed them.
\newline

\noindent
In addition, my experience as a PhD learned me how to solve problems, and hence
be proactive which I think is a good quality when working in the 
Research, Design and Development department. 
For instance, one issue that I met was 
post-processing: the simulations that are ran today at CERFACS are so big
that the size of the databases is not easy to handle and hence to process. 
No acceptable solution was available and I decided to develop my own post-processing software 
called Antares (www.cerfacs.fr/antares). As I have always been convinced that
the tips and tools that makes our life easier at work should be shared, 
a full documentation has been written as well as a website with an on-line tutorial.
When the tool became mature, we promoted it to industrial partners. For now
the tool is used by research companies worldwide (France, Belgium and Canada).
I am fully aware that this problem was not directly related to my PhD work but
this issue prevented us from deeply analysing our simulations and hence understanding
the phenomena that develop in it. This experience taught me how
to develop a product for others so that their work can become easier, which I have
understand to be the essence of Dyson strategy.
\newline

\noindent
This is why, I thoroughly appreciate Dyson way of thinking as I believe that
a good product is a product that simply works. In our time, lots
of companies misspend their money to build products that are of 
bad quality so that they can be cheap enough, 
thinking that people will buy new ones each year, which I
think is a defeatist way of working. 
In opposite, Dyson has understand that there is plenty of room
for a company that does its job well with high-quality, great-designed and 
highly-technological products. This is why I hope to have
the opportunity to bring my small contribution to this great goal.
\newline

\noindent
I am available in March 2014 and I am seeking 
a full time job. I would be pleased to give you
any further information in an interview at any time which is convenient to you.
I look forward to hearing from you.
\\*
\\*
Yours faithfully, \\
\\*
\\*
Adrien Gomar \\
\\*
\\*
Enc. : curriculum vitae

\end{document}