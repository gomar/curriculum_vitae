\documentclass[10pt]{article}

\usepackage{mycv}

% ================
% = THE DOCUMENT =
% ================
\begin{document}
\begin{tabular}[ht]{clr}
	\multicolumn{2}{l}{\LARGE Adrien Gomar} & {\large available on March 2014} \\
	& & \phantom{aaaaaaaaaaaaaaaaaaaa} {references available upon request} \\
	\fasymbol{"F041} & 1 allée Charles Malpel, Appt. A72, 31300 Toulouse, France \\
	\fasymbol{"F095} & (+33) 6 14 32 24 24 \\
	\fasymbol{"F0E0} & \href{mailto:adrien.gomar@gmail.com}{adrien.gomar@gmail.com} \\
	\multicolumn{2}{l}{\url{http://cerfacs.fr/~gomar}} & \\
\end{tabular}

% ==========
% = SKILLS =
% ==========
\mysec{Skills \& Abilities}
\begin{mytable}
\tblentry{Summary}{

- Forthcoming \underline{Ph.D. in Computational Fluid Dynamics} (CFD) with application to unsteady compressible aerodynamics. Defense expected in February 2014 \newline
- \underline{Master degree in Aerospace Engineering} \newline
- \underline{Advanced level in English}: wrote reports and publications related to my work, made
technical presentations and animated meetings \spacerows \\}

\tblentry{Computer skills}{

- Used several software program dedicated to \underline{CFD}: 
\href{http://elsa.onera.fr}{\textit{elsA}}, 
\href{http://www.numeca.com/en/products/finetmturbo}{Fine/Turbo}
and \href{http://www.ansys.com/Products/Simulation+Technology/Fluid+Dynamics/Fluid+Dynamics+Products/ANSYS+Fluent}{Fluent} \newline
- Proficient in \underline{Python}, NumPy, Scipy. 
Ability with \underline{object oriented} programming languages. 
Developed Antares post-processing library (\href{http://cerfacs.fr/antares/}{www.cerfacs.fr/antares}) \newline
- Used \underline{Git} control version system for Antares and my daily workflow \spacerows \\}

\tblentry{Communication}{

- Promulgated Antares software program:
\underline{designed a website} (\href{http://cerfacs.fr/antares/}{www.cerfacs.fr/antares})
with tutorials and \underline{promoted it to industrial partners} (Airbus, Snecma, Turbomeca) \newline
- \underline{Presentation skills}: made several technical presentations for a large audience (>30 people), 
skillful use of design rules (color schemes, diagrams instead of text, clarity of slides)\spacerows \\}

\tblentry{Leadership skills}{

Developed my leadership skills through a \underline{team leader position} at Mac Donald's (team of 15 people)}
\end{mytable}

% ==============
% = EXPERIENCE =
% ==============
\mysec{Experience}
\begin{mytable}
\tblentry{2011 \\ up to now}{

{\headingfont CERFACS, Toulouse, France, 
Doctoral thesis at numerical computation center}


entitled "Counter rotating open rotor aeroelasticity", I currently: \newline
- extend a reduce order method for the unsteady weak coupling of aeroelasticity in turbomachines 
within an industrial CFD code (\href{http://elsa.onera.fr}{\textit{elsA}}), \newline
- validate the model against experimental data on a reference configuration, \newline
- explain the limitations of the model and improve its robustness for industrial applications, \newline
- apply the method on an industrial contra-rotating open rotor configuration. \newline
- develop a python post-processing library (\href{http://cerfacs.fr/antares/}{www.cerfacs.fr/antares})
to ease the process for large simulation databases. 
Promote it: development of a website and 
presentations to industrial partners (Airbus, Snecma, Turbomeca) and train related. \newline
- teach fluid dynamics at ISAE (graduate school of Aerospace Engineering) \spacerows \\}

\tblentry{2010 \\ (6 months)}{

{\headingfont CERFACS, Toulouse, France, 
Engineering internship at numerical computation center}

Computed Large Eddy Simulation of a high-pressure compressor stage.
Post-processed and analyzed large simulations (>Go).
Compared the results to experimental data provided by the NASA.
Wrote proceeding for 9\textsuperscript{th} European Turbomachinery Conference \spacerows \\}

\tblentry{2009 \\ (3 months)}{

{\headingfont LIEBHERR Aerospace, Toulouse, France, 
Engineering internship at turbomachinery department}

Benchmarked two CFD software programs: \textit{elsA} 
and Fine/Turbo on a centrifugal compressor \spacerows \\}

\tblentry{2008-2009}{

{\headingfont Institut Supérieur de l'Aéronautique et de l'Espace (ISAE), Toulouse, France, 
Long student project}

Set an experimental approach to analyze an airplane extract fan. 
Benchmarked experimental and simulation results (using \href{http://www.numeca.com/en/products/finetmturbo}{Fine/Turbo}). 
Presented the results to an industrial partner (Technofan) \\}

\multicolumn{2}{c}{\phantom{a}} \\

\tblentry{2007-2009 \\ (2.5 years)}{

{\headingfont Mac Donald's, Toulouse, France}

Team leader. Trained new recruits to Mac Donald's standards. 
Managed a team of 15 people with good interpersonal skills and team spirit}
\end{mytable}

% =============
% = EDUCATION =
% =============
\mysec{Education}
\begin{mytable}
\tblentry{Feb 2014 (expected)}{

{\headingfont Arts et Métiers ParisTech, 
Paris, France}

PhD in Computational Fluid Dynamics with application 
to unsteady compressible aerodynamics
\spacerows \\}

\tblentry{Sept 2010}{

{\headingfont Institut Supérieur de l'Aéronautique et de l'Espace (ISAE), 
Toulouse, France}

MS in aeronautical engineering}
\end{mytable}

% % ================
% % = PUBLICATIONS =
% % ================
% \mysec{Publications}
% \begin{center}
% 	\begin{minipage}[ht]{0.9\textwidth}
% 		\nocite{Guedeney2013317}
% 		\nocite{Sicot2012uq}
% 		\bibliography{publication}
% 	\end{minipage}
% \end{center}


\bibliographystyle{plain}
\end{document}